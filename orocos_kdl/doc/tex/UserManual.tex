\documentclass[a4paper,10pt]{report}

\title{User Manual \\ Kinematics \& Dynamics Library}
\author{Ruben Smits, Herman Bruyninckx}

\usepackage{listings}
\usepackage{amsmath}
\usepackage{url}
\usepackage{todo}
\usepackage{hyperref}

\begin{document}
\lstset{language=c++}

\maketitle

\chapter{Introduction to KDL}
\label{cha:introduction-kdl}

\section{What is KDL?}
\label{sec:what-kdl}
Kinematics \& Dynamics Library is a c++ library that offers

\begin{itemize}
\item classes for geometric primitives and their operations
\item classes for kinematic descriptions of chains (robots)
\item (realtime) kinematic solvers for these kinematic chains
\end{itemize}

\section{Getting support - the Orocos community}
\label{sec:gett-supp-oroc}

\begin{itemize}
\item This document!
\item The website : \url{http://www.orocos.org/kdl}.
\item The mailinglist. \todo{add link to mailinglist}
\end{itemize}


\section{Getting Orocos KDL}
\label{sec:getting-orocos-kdl}

First off all you need to successfully install the KDL-library. This is
explained in the \textbf{Installation Manual}, it can be found on
\url{http://www.orocos.org/kdl/Installation_Manual}


\chapter{Tutorial}
\label{cha:tutorial}

\paragraph{Important remarks}
\label{sec:important-remarks}
\begin{itemize}
\item All geometric primitives and there operations can be used in
  realtime. None of the operations lead to dynamic memory allocations
  and all of the operations are deterministic in time.
\item All values are in [m] for translational components and [rad] for
  rotational components.
\end{itemize}

\section{Geometric Primitives}
\label{sec:geometric-primitives}

\paragraph{Headers}
\label{sec:headers}

\begin{itemize}
\item \lstinline$frames.hpp$: Definition of all geometric primitives
  and there transformations/operators.
\item \lstinline$frames_io.hpp$: Definition of the input/output
  operators.
\end{itemize}


The following primitives are available:
\begin{itemize}
\item Vector~\ref{sec:vector}
\item Rotation~\ref{sec:rotation}
\item Frame~\ref{sec:frame}
\item Twist~\ref{sec:twist}
\item Wrench~\ref{sec:wrench}
\end{itemize}

\subsection{Vector}
\label{sec:vector}
Represents the 3D position of a point/object. It contains
three values, representing the X,Y,Z position of the point/object with
respect to the reference frame:

$\mathrm{Vector} = \left[\begin{array}{c} x \\ y \\ z \end{array} \right]$

\paragraph{Creating Vectors}
\label{sec:creating-vectors}
There are different ways to create a vector:
\begin{lstlisting}
  Vector v1; //The default constructor, X-Y-Z are initialized to zero
  Vector v2(x,y,z); //X-Y-Z are initialized with the given values
  Vector v3(v2); //The copy constructor
  Vector v4=Vector::Zero(); //All values are set to zero
\end{lstlisting}

\paragraph{Get/Set individual elements}
\label{sec:gets-indiv-elem}
The operators \lstinline$[]$ and \lstinline$()$ use indeces from 0..2,
index checking is enabled/disabled by the DEBUG/NDEBUG definitions:
\begin{lstlisting}
  v1[0]=v2[1];//copy y value of v2 to x value of v1 
  v2(1)=v3(3);//copy z value of v3 to y value of v2
  v3.x( v4.y() );//copy y value of v4 to x value of v3
\end{lstlisting}

\paragraph{Multiply/Divide with a scalar}
\label{sec:mult-with-scal}
You can multiply or divide a Vector with a double using the operator
\lstinline$*$ and \lstinline$/$:
\begin{lstlisting}
  v2=2*v1;
  v3=v1/2;
\end{lstlisting}

\paragraph{Add and subtract vectors}
\label{sec:add-subtract-vectors}
You can add or substract a vector with another vector:
\begin{lstlisting}
  v2+=v1;
  v3-=v1;
  v4=v1+v2;
  v5=v2-v3;
\end{lstlisting}

\paragraph{Cross and scalar product}
\label{sec:cross-scalar-product}
You can calculate the cross product of two vectors, which results in
new vector or calculate the scalar(dot) product of two vectors:
\begin{lstlisting}
  v3=v1*v2; //Cross product
  double a=dot(v1,v2)//Scalar product
\end{lstlisting}

\paragraph{Resetting}
\label{sec:resetting}
You can reset the values of a vector to zero:
\begin{lstlisting}
  SetToZero(v1);
\end{lstlisting}

\paragraph{Comparing vectors}
\label{sec:comparing-vectors}
With or without giving an accuracy:
\begin{lstlisting}
  v1==v2;
  v2!=v3;
  Equal(v3,v4,eps);//with accuracy eps
\end{lstlisting}

\subsection{Rotation}
\label{sec:rotation}

Represents the 3D rotation of an object wrt the reference
frame. Internally it is represented by a 3x3 matrix which is a
non-minimal representation:

$\mathrm{Rotation} =
\left[\begin{array}{ccc} 
    Xx&Yx&Zx\\
    Xy&Yy&Zy\\
    Xz&Yz&Zz
  \end{array}\right]
$

\paragraph{Creating Rotations}
\label{sec:creating-rotations}

\subparagraph{Safe ways to create a Rotation}
\label{sec:safe-ways-create}

The following result always in consistent Rotations. This means the
rows/columns are always normalized and orthogonal:

\begin{lstlisting}
  Rotation r1; //The default constructor, initializes to an 3x3 identity matrix
  Rotation r1 = Rotation::Identity();//Identity Rotation = zero rotation
  Rotation r2 = Rotation::RPY(roll,pitch,yaw); //Rotation build out off Roll-Pitch-Yaw angles
  Rotation r3 = Rotation::EulerZYZ(alpha,beta,gamma); //Rotation build out off Euler Z-Y-Z angles
  Rotation r4 = Rotation::EulerZYX(alpha,beta,gamma); //Rotation build out off Euler Z-Y-X angles
  Rotation r5 = Rotation::Rot(vector,angle); //Rotation build out off an equivalent axis(vector) and an angle.
\end{lstlisting}

\subparagraph{Other ways}
\label{sec:other-ways}
The following should be used with care, they can result in
inconsistent rotation matrices, since there is no checking if
columns/rows are normalized or orthogonal
\begin{lstlisting}
  Rotation r6( Xx,Yx,Zx,Xy,Yy,Zy,Xz,Yz,Zz);//Give each individual element (Column-Major)
  Rotation r7(vectorX,vectorY,vectorZ);//Give each individual column
\end{lstlisting}

\paragraph{Getting values}
\label{sec:getting-values}
Individual values, the indices go from 0..2: 

\subsection{Frame}
\label{sec:frame}

\subsection{Twist}
\label{sec:twist}

\subsection{Wrench}
\label{sec:wrench}


\section{Kinematic Structures}
\label{sec:kinematic-structures}

\subsection{Joint}
\label{sec:joint}

\subsection{Segment}
\label{sec:segment}

\subsection{Chain}
\label{sec:chain}

\section{Kinematic Solvers}
\label{sec:kinematic-solvers}

\subsection{Forward position kinematics}
\label{sec:forw-posit-kinem}

\subsection{Forward velocity kinematics}
\label{sec:forw-veloc-kinem}

\subsection{Jacobian calculation}
\label{sec:jacobian-calculation}

\subsection{Inverse velocity kinematics}
\label{sec:inverse-veloc-kinem}

\subsection{Inverse position kinematics}
\label{sec:inverse-posit-kinem}

\section{Motion Specification Primitives}
\label{sec:moti-spec-prim}

\subsection{Path}
\label{sec:path}

\subsection{Velocity profile}
\label{sec:velocity-profile}

\subsection{Trajectory}
\label{sec:trajectory}


\end{document}

%%% Local Variables: 
%%% mode: latex
%%% TeX-master: t
%%% End: 
